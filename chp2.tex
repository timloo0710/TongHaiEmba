\chapter{文獻探討}


\section{虛擬社群}

本節旨在介紹虛擬社群,首先整理過去學者對虛擬社群的定義,以了解虛擬
社群的意涵與特徵,接著介紹虛擬社群由哪些要素組成,了解這些組成要素對虛
擬社群的重要性,最後介紹虛擬社群的種類,探討不同類型的社群如何滿足不同
成員的需求。


\section{虛擬社群的定義}
「虛擬社群」是由「實體社群」的概念演化而來的。實體社群是指一群人因
具有共同的興趣或特徵而於現實生活中聚集在一起的組織。相對於實體社群,
「虛
擬」兩字點出虛擬社群的無形性,社群成員不一定會在現實中實際面對面團聚在
一起,但透過電腦科技的輔助,成員們能在網路空間組織社會關係(Lu et al., 2010)。
虛擬社群消弭了時間與距離的限制,使得成員能夠在社群中結交志同道合的朋友,
特別是這些網路上的朋友並不是他們在現實生活中有機會能認識的(Wang, Yu, &
Fesenmaier, 2002),虛擬社群提供一個平台讓這些原本不認識的人能夠在網路世
界相遇、相識,在此平台內彼此可透過社會互動產生連結,也藉由此連結關係互
相交換資源(Chiu, Hsu, & Wang, 2006)。
虛擬社群是個涵蓋多學科的概念,由不同觀點來解釋虛擬社群會有不同的定

義(Gupta & Kim, 2004; Preece, 2000, 2001),故以下分別由科技、社會學、商業及
綜合多學科的層面來探討虛擬社群的定義。
由科技的角度來檢視,Lu et al. (2010)認為虛擬社群是以電腦設備及資訊科
技為技術基礎,促使社群成員能在網路空間彼此互相溝通、互動與發展人際關係,
而社群的內容主要由成員共同創作產生。Wasko, Teigland, and Faraj (2009)也指出
虛擬社群是一個使用電腦媒介來創造與維持的溝通系統,具備自我組織、自主、
開放參與的特性。換句話說,當身處任何地點的人們能夠透過以電腦為媒介的通
訊技術經常性地互相溝通時,這群人必然會形成一個社群(Rheingold, 1993)。
以社會學的觀點來探討,Fernback and Thompson (1995)認為虛擬社群是在網
路世界的特定邊界下,透過持續、反覆的互動而形成的社會關係。Lu and Yang
(2011)定義虛擬社群是一種社會網絡關係,成員透過互動來達到分享資訊、知識
與從事社會交往的目的,而社會互動及鑲嵌在社會網絡裡的資源是維持虛擬社群
的重要因素。
由商業角度來探討,Dholakia, Bagozzi, and Pearo (2004)認為可將虛擬社群視
為不同大小的消費群體於網路上相識與互動,以達到個人以及社群成員的共同目
標。
由綜合層面來定義,Koh and Kim (2004)將虛擬社群定義為一群具有共同興
趣或目標的人,因存在對知識或資訊的需求而於網路空間互動。Chiu, Hsu, and
Wang (2006)也提出類似的觀點,認為成員不同於一般網路使用者,成員是因為
共同的興趣、目標、需求或習慣而聚集在一起。


由上述多位學者的定義,我們可以歸納出虛擬社群具有以下幾點特徵:
(1) 由一群人組成。
(2) 成員因共同的興趣與目標而凝聚在一起。
(3) 存在於網路空間。
(4) 透過資訊科技互相溝通與互動。
(5) 內容是由社群成員協同創作產生。
(6) 允許成員在社群中發展社會關係。
可見虛擬社群為一群人因興趣、習慣、需求、目標等原因而聚集在某個網路
平台,藉由電腦設備與資訊科技的輔助,彼此可在該平台上互相溝通與交流,經
長時間持續、反覆的互動而形成的社會網絡關係。






\section{}

